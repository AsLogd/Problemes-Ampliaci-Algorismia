\section{Problema examen 4}
\subsection{Enunciat}
\textbf{Maximizing the level of joint initiative}. We have a social network modeling the interaction among members of a company. The social situation is modeled by an node weighted undirected graph. The manager keeps as node weights the \textit{level of individual initiative} of the member which is a non negative number measuring the ability of working by itself successfully in a project. The edges represents interactions among members that have resulted in a clear decrease of effectiveness when working together in some project. The manager wants to select a group of people to form a team to develop a new project. For doing so, the company defines the \textit{level of joint initiative of a group} as the sum of the level of individual initiative of each member in the selected group divided by the sum of the level of individual initiative of all the members of the company. The goal of the manager is to find a team having maximum level of joint initiative among those sets in which no pair of persons has shown a clear decrease of effectiveness when working together in some project.

\vspace{5mm}
(i) Provide a formalization of the problem. 

\vspace{5mm}
(ii) Show that the problem parameterized by the treewidth of the graph belongs to FPT.

\subsection{Solució}