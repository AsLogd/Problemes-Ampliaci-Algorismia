\section{Problema 27}
\subsection{Enunciat}
Show that the following parameterized edge deletion problem belongs to FPT.
\par
\textbf{TRIANGLE ELIMINATION($G$, $k$)}: $G$ is a graph and $k\geq0$ an integer.
\par
Parameter: $k$
\par
Question: Does it exist an edge subset $F$ so that $|F| \leq k$ and $G-F$ has no triangles?

\subsection{Solució}
Per demostrar que el problema pertany a FPT, hem de donar un algorisme amb cost $O(f(k)*|x|^{O(1)})$.
En aquest cas farem servir la tècnica del \textit{bounded search tree} limitat a $k$ passos.
\par
\textbf{TRI-ELIM-BST($G=\{V,E\}$, $k$)}: $G$ is a graph and $k\geq0$ an integer.


\begin{algorithm}
\caption{TRI-ELIM-BST}\label{euclid}
\begin{algorithmic}[1]
\Function{getTriangle}{$e$} \Comment{If e forms a triangle, returns the other 2 edges}
\State{($v_s$, $v_e$) $\gets$ $e$}
\State{$l \gets \emptyset$}
\For{$v_{sn}$ in neighbours($v_s$)}
    \For{$v_{en}$ in neighbours($v_e$)}
        \If{$v_{sn} == v_{en}$}
            \State $e_2 \gets (v_s, v_{sn})$
            \State $e_3 \gets (v_e, v_{sn})$
            \State \Return $(e_2, e_3)$
        \EndIf
    \EndFor
\EndFor
\State \Return null
\EndFunction
\newline
\Function{TRI-ELIM-BST}{$G=\{V,E\}$, $k$}
\If{$k >= |E|-2$}
    \State \Return true
\EndIf
\If{$k < 0$}
    \State \Return false
\EndIf
\For{$e_1$ in $E$}
    \State $l \gets getTriangle(e_1)$
    \If{$l$ is not null}
        \State $(e_2, e_3) \gets l$
        \State {\#Graph without triangles found}
        \State {found $\gets$ TRI-ELIM-BTS($G-e_1$, $k-1$)} 
        \State {found $\gets$ found  $||$ TRI-ELIM-BTS($G-e_2$, $k-1$)}
        \State {found $\gets$ found  $||$  TRI-ELIM-BTS($G-e_3$, $k-1$)}
        \State \Return found
    \Else
        \State $G \gets G - e_1$
    \EndIf
\EndFor
\State \Return{true} \Comment{If no triangle found, return true}
\EndFunction
\end{algorithmic}
\end{algorithm}
\subsubsection{Correctesa}
Analitzem la correctesa de l'algorisme \textbf{TRI-ELIM-BST}:
\begin{itemize}
    \item En el cas que tinguem $k > |E|-2$, es trivial veure que si extraiem totes les arestes menys dos, no poden existir triangles a $G$. Per tant retornem cert.
    \item En el cas en que arribem a $k < 0$, ja no podem fer mes passos, per tant retornem fals.
    \item Per cada aresta de $E$ existeixen dos casos:
     \begin{itemize}
            \item La aresta forma almenys un triangle: En aquest cas, hem de treure almenys una aresta per eliminar el triangle. Com que no sabem quina aresta ens convé treure, provem a treure les tres. En cada cas, restem $k$ en 1, ja que són arestes que formarien part de $F$. Suposant que la crida recursiva es correcte, si en algun dels casos retorna cert, vol dir que seguint aquell camí de l'arbre de solucions hem arribat a una solució, per tant retornem cert.
            \item La aresta no forma cap triangle: en aquest cas, aquesta aresta mai formarà un triangle, ja que només extraiem arestes. Aquestes arestes no formen part de $F$, per tant les podem treure sense alterar $k$.
        \end{itemize}
    \item En cas de no trobar cap triangle, retornem cert. Hem aconseguit obtenir un graf sense triangles.
\end{itemize}
\newpage
\subsubsection{Complexitat}
Per analitzar la complexitat de l'algorisme aplicarem el teorema mestre:
\begin{itemize}
    \item $a = 3$. 3 Crides recursives
    \item $n = k$. Paràmetre que disminueix
    \item $c = 1$. Quantitat de disminució
\end{itemize}
Per tant
\[
a > 1 \Rightarrow T(k) \in 3^k \Rightarrow \text{TRI-ELIM-BST} \in O(3^k)
\]
