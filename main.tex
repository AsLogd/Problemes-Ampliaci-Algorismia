\documentclass{article}
\usepackage[utf8]{inputenc}

\usepackage{amssymb}
\usepackage{mathtools}
\usepackage{amsthm}


\title{Ampliació Algorísmia}
\date{April 2018}

\begin{document}

\maketitle

\section{Problema 3}
\subsection{Enunciat}
Consider the \textbf{MAXIMUM COVERAGE} problem: Given sets $S_1,...,S_m$ over a universe of elements $U = \{1,...n\} = \bigcup_{i=1}^{n} S_i$ and a positive integer k. Choose k sets that cover as many elements as possible, that is \[
opt(x) = \text{max}\{|\bigcup_{i \in I} S_i| : |I| = k\}
\]
Provide a greedy approximation algorithm with rate $\frac{e}{e-1} \approx 1.58$
\subsection{Solució}
Primer hem de proporcionar un algorisme. Una bona idea es intentar amb un \textit{greedy} basant-nos en algun heurístic i després veure si podem demostrar la garantia d'aproximació.
En aquest cas proposem el següent algorisme:
\newline
\par
\textbf{NEXT COVER($S$, $k$)}: Escollir el set de $S$ amb més elements sense cobrir. Marcar els nous elements com a coberts. Repetir $k$ vegades.
\newline
\newline
Hem de demostrar el següent:
\[
\text{OPT}*\frac{e-1}{e} \leq m \leq \text{OPT}
\]
on $m$ és la solució donada per \textbf{NEXT COVER}.
\newpage
\subsubsection{Demostració}
Definim $D_i$ com la diferència entre el número d'elements coberts a la solució òptima OPT i el número d'elements coberts a la solució parcial ($c_i$) obtinguda al pas $i$:
\[
D_i = OPT-c_i
\]
\par
\textbf{Lema a}: Al pas $i+1$, cobrim, com a mínim, $\frac{D_i}{k}$ nous elements. Per tant:
\[
D_{i+1} = D_i - \frac{D_i}{k}
\]
\textbf{Demostració lema a:}
\newline
Sabem que OPT és el número de elements coberts seleccionant $k$ conjunts de l'entrada, per tant, si a la primera iteració ($D_0=\text{OPT}$) seleccionem el conjunt més gran (el conjunt que més elements nous té), sabem que com a mínim estem cobrint $c_1=\frac{OPT}{k}$ elements nous 
\footnote{Això es una aplicació del \textit{pigeonhole principle}. Un parell d'arguments per convèncer-nos de que això es correcte: \newline -$\frac{\text{OPT}}{k}$ es el cas en el que el conjunt més gran de la solució òptima te mida mínima. \newline - Si el conjunt més gran tingués menys de $\frac{\text{OPT}}{k}$ elements, seria impossible arribar a OPT en $k$ passos, ja que la resta de conjunts contindrien encara menys elements\newline} (per tant, $D_1=D_0-c_1=OPT-\frac{OPT}{k}$).
\newline
Al següent pas, poden passar dues coses:
\begin{itemize}
    \item cas a: Al pas 1, hem seleccionat un conjunt de la solució òptima (amb $c_1$ elements), per tant, utilitzant l'argument anterior, sabem que al pròxim pas podrem agafar, com a mínim, $\frac{OPT-c_1}{k-1}=\frac{D_1}{k-1}$ elements nous (es a dir, els elements que falten dividit entre $k-1$ passos).
    \item cas b: Al pas 1, hem seleccionat un conjunt que no és de la solució òptima (amb $c_1$ elements). En aquest cas només podem assegurar que al pròxim pas podrem agafar $\frac{OPT-c_1}{k}=\frac{D_1}{k}$ elements nous, ja que no hem progressat. En altres paraules, hem perdut un "torn" escollint un conjunt equivocat    \footnote{Cal notar que al seleccionar conjunts que no formen part de la solució òptima, en cap cas podem arribar a cobrir més elements que OPT. Sense pèrdua de generalitat, podem afirmar que si seleccionem un conjunt que no forma part de la solució optima, aconseguirem un resultat pitjor: $m < \text{OPT}$}.
\end{itemize}
De forma similar, al pas $i$ tenim els dos mateixos casos, però tenint en compte que podem restar a la $k$ el número de conjunts $x$ que formen part de la solució òptima que hem acumulat a la nostre solució parcial.
\begin{itemize}
    \item cas a: $\frac{OPT-c_i}{k-x-1}=\frac{D_i}{k-x-1}$
    \item cas b: $\frac{OPT-c_i}{k-x}=\frac{D_i}{k-x}$
\end{itemize}

En qualsevol cas, i en qualsevol pas $i$, sempre podem assegurar, com a mínim, que podem cobrir $\frac{D_i}{k}$ elements nous (en el cas pitjor, no encertem cap conjunt de la solució òptima $x=0$).
\qed

Pel \textbf{lema a}, sabem:
\begin{eqnarray}
    D_k \leq D_{k-1}-\frac{D_{k-1}}{k} = D_{k-1}(1-\frac{1}{k})\\
    \leq D_{k-2}(1-\frac{1}{k})^2 \nonumber\\
    ... \nonumber \\
    \leq D_{0}(1-\frac{1}{k})^k = OPT*(1-\frac{1}{k})^k \\
    = OPT* \frac{1}{e} \\
    \Rightarrow D_k \leq OPT* \frac{1}{e} \nonumber
\end{eqnarray}
\par
(1) La diferència a l'últim pas $k$ es més petita que la diferència al pas anterior $k-1$ menys $\frac{D_{k-1}}{k}$, ja que sabem que hem agafat aquesta quantitat d'elements nous o més; Extraiem el factor comú
\newline
\par
(2) Repetint aquest argument $k$ vegades arribem al pas 0, on $D_0 = \text{OPT}$
\newline
\par
(3) $\frac{1}{e}=\lim_{k \longrightarrow \inf}(1-\frac{1}{k}^k)$
\newline
\newline
Finalment,
\begin{eqnarray}
    D_k = OPT - m \leq OPT*\frac{1}{e} \nonumber\\ 
    \Rightarrow m \geq -OPT*\frac{1}{e}+OPT \nonumber\\
    = OPT*(1-\frac{1}{e}) \nonumber\\
    = OPT*\frac{e-1}{e}  \nonumber\\
    \Rightarrow OPT*\frac{e-1}{e} \leq m \leq OPT \nonumber
\end{eqnarray}
\renewcommand\qedsymbol{$\blacksquare$}
\qed

\end{document}
